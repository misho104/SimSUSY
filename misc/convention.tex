\documentclass[a4paper,11pt,captions=tableheading,DIV=12]{scrartcl}
\pdfoutput=1

% ----------------------------------------------------------- Packages
\usepackage{amsmath,amssymb,url,cite,slashed,cancel,booktabs,graphicx,listings,subcaption}
\usepackage[colorlinks=true,urlcolor=blue,citecolor=magenta]{hyperref}
%%%UNUSED%%% \usepackage{feynmp,enumerate,multirow,wrapfig}
\renewcommand\citepunct{,\penalty1000\hskip.13emplus.1emminus.1em\relax} % no line-break in \cite

% MATH NOTATION
\newcommand\w[1]{_{\mathrm{#1}}}
\newcommand\vc[1]{{\boldsymbol{#1}}}
\newcommand\dd{\mathrm{d}}
\newcommand\ee{\mathrm{e}}
\newcommand\ii{\mathrm{i}}
\newcommand\pmat[1]{\begin{pmatrix}#1\end{pmatrix}}
\newcommand\Order{\mathop{\mathcal{O}}}
\newcommand\Br{\mathop{\mathrm{Br}}}
\newcommand{\dn}[3]{\frac{\dd^#1 #2}{\dd #3^#1}}
\newcommand{\pdn}[3]{\frac{\partial^#1 #2}{\partial #3^#1}}
\newcommand{\pd}[2]{\frac{\partial #1}{\partial #2}}

\newcommand\unit[1]{\,\mathrm{#1}}
\newcommand\eV{\unit{eV}}
\newcommand\keV{\unit{keV}}
\newcommand\MeV{\unit{MeV}}
\newcommand\GeV{\unit{GeV}}
\newcommand\TeV{\unit{TeV}}
\newcommand\fb{\unit{fb}}
\newcommand\pb{\unit{pb}}
\newcommand\iab{\unit{ab^{-1}}}
\newcommand\ifb{\unit{fb^{-1}}}
\newcommand\ipb{\unit{pb^{-1}}}
\newcommand\fm{\unit{fm}}

\makeatletter
\def\EE{\@ifnextchar-{\@@EE}{\@EE}}
\def\@EE#1{\ifnum#1=1 \times\!10 \else \times\!10^{#1}\fi}
\def\@@EE#1#2{\times\!10^{-#2}}
\makeatother

\newcommand{\fracp}[3]{\ifnum1=#3\relax\biggl(\frac{#1}{#2}\biggr)\else\biggl(\frac{#1}{#2}\biggr)^{#3}\fi}
\newcommand{\neut}[1][]{{\tilde\chi^0_{#1}}}
\newcommand{\chrg}[1][]{{\tilde\chi^\pm_{#1}}}
\newcommand{\chrgP}[1][]{{\tilde\chi^+_{#1}}}
\newcommand{\chrgM}[1][]{{\tilde\chi^-_{#1}}}

\lstset{columns=[l]fullflexible,basicstyle=\small\ttfamily,xleftmargin=2em,frame=L,keepspaces=true}
\bibliographystyle{utphys27mod}

% ---------------------------------------------------- For Sho's Notes
\usepackage{scrlayer-scrpage,color,soul}
\usepackage[hhmmss]{datetime}
\newdateformat{mydate}{\THEDAY\;\shortmonthname.\;\THEYEAR}
\addtokomafont{pagehead}{\small\normalfont}
\rohead*{\texttt{[\jobname~@~\mydate\today~\currenttime]}}

\newcommand{\TODO}[1]{\begingroup\color{red}\textbf{$\clubsuit$#1$\clubsuit$}\endgroup}

\title{SLHA convention compared against References}
\author{Sho Iwamoto}
\date{}
\newcommand{\wL}{_{\mathrm L}}
\newcommand{\wR}{_{\mathrm R}}
\newcommand{\bU}{\bar U}
\newcommand{\bD}{\bar D}
\newcommand{\bE}{\bar E}
\newcommand{\tra}{^{\mathrm T}}
\begin{document}
\maketitle
%---------------------------------------------------------------------

\subsection{Higgs potential}

\subsubsection{SLHA convention}
The SLHA~\cite{SLHA} is based on Gunion--Haber's notation~\cite{GH}:
\begin{align}
\begin{split}
  W &\supset-\epsilon_{ab}
 \mu H_1^aH_2^b
 \\&=-\mu(H_1^1H_2^2-H_1^2H_2^1)
 \\&\equiv \mu(-H_1^0H_2^0+H_1^-H_2^+),
\end{split}
 \tag{SLHA:3}\\
\begin{split}
  V_2 &\supset
  m_{H_{1}}^{2}|H_1|^2
 + m_{H_{2}}^{2}|H_2|^2
 - (m_3^2\epsilon_{ab}H_1^aH_2^b +\text{h.c.}),
\\
&=
  m_{H_{1}}^{2}|H_1|^2
 + m_{H_{2}}^{2}|H_2|^2
+ \left[m_3^2 (-H_1^0H_2^0+H_1^-H_2^+) +\text{h.c.}\right].
\end{split}
 \tag{SLHA:7}
\end{align}
where $\epsilon_{12}=\epsilon^{12}=+1$.
The parameter $m_A^2$, set by \texttt{EXTPAR 24}, is then defined as
\begin{equation}
 m_A^2 = \frac{2m_3^2}{\sin2\beta}.
\tag{SLHA:8}
\end{equation}

\subsubsection{Comparison to GH convention}
In Gunion--Haber~\cite{GH}, the definitions are as follows, with their errata applied:
\begin{align}
 W &\supset -\epsilon_{ab}\mu H_1^aH_2^b,\tag{GH:3.3--3.4}
\\
V\w{soft}&\supset m_1^2|H_1|^2 + m_2^2 |H_2|^2 - (m_{12}^2\epsilon_{ab}H_1^a H_2^b+\text{h.c.}),
\tag{GH:3.9}
%
\end{align}
where $\bE\equiv\tilde R$ in their notation, with the same definition $\epsilon_{12}=1$ (found below Eq.~(3.2)).
So, with the identification $(m_1^2,m_2^2,m_{12}^2)\equiv(m_{H_1}^2,m_{H_2}^2,m_3^2)$, this is identical to SLHA convention.
Also we note
\begin{align}
 \langle H_1\rangle &= \pmat{v_1\\0},&
 \langle H_2\rangle &= \pmat{0\\v_2},
\tag{GH:3.7}
\end{align}
with $v_1>0$, $v_2>0$ (3.24), and $\tan\beta=v_2/v_1$ (2.8).

We can then use their results, which say
\begin{align}
 &m_{H_1}^2 = -|\mu|^2 + 2\lambda_1 v_2^2 - m_Z^2/2,
\tag{GH:3.21c}
\\
 &m_{H_2}^2 = -|\mu|^2 + 2\lambda_1 v_1^2 - m_Z^2/2,
\tag{GH:3.21d}
\\
 &m_{H_1}^2 + m_{H_2}^2 + 2|\mu|^2 
= m_{3}^2(\tan\beta+\cot\beta)
= \frac{2m_{3}^2}{\sin2\beta},
\tag{GH:3.22}
\end{align}
and, noting that
\begin{align}
 m_Z^2&=(g^2+g'^2)(v_1^2+v_2^2)/2,&
 m_W^2&=g^2(v_1^2+v_2^2)/2,
\end{align}
which are found below Eq.~(3.19),
\begin{equation}
 \lambda_1
= \frac{g^2+g'^2}{4} + \frac{m_3^2}{2v_1v_2}
= \frac{m_Z^2}{2(v_1^2+v_2^2)} + \frac{m_3^2}{2v_1v_2};
\end{equation}
combining them, we have
\begin{align}
 m_{H_1}^2&=-|\mu|^2-\frac{m_Z^2}{2}\cos2\beta + m_3^2\tan\beta,\\
 m_{H_2}^2&=-|\mu|^2+\frac{m_Z^2}{2}\cos2\beta + m_3^2\cot\beta.
\end{align}
Also the Higgs parameters are given by
\begin{align}
 m_{H^\pm}^2&=(4\lambda_1-g'^2)(v_1^2+v_2^2)=m_W^2+m_A^2,\tag{GH:3.16}\\
 m_{H_3^0}^2&=m_{H^\pm}^2-m_W^2=m_A^2,\tag{GH:3.17}\\
 m_{H_1^0,H_2^0}^2 &=
 \frac12\left[m_{H_3^0}^2+m_Z^2\pm\sqrt{
 (m_{H_3^0}^2+m_Z^2)^2-4m_Z^2 m_{H_3^0}^2\cos^22\beta
}\right],\tag{GH:3.18}\\
\tan2\alpha&=\frac{m^2_{H_3^0}+m_Z^2}{m_{H_3^0}^2-m_Z^2}\tan2\beta\tag{GH:3.19}.
\end{align}

\subsubsection{Comparison to SUSY Primer convention}
We here compare the SLHA/GH notation with Martin's SUSY primer.
The potentials are given by
\begin{align}
 W\w{MSSM}
&\supset
 \mu H\w u H\w d 
\equiv
\mu (H_2^+H_1^- - H_1^0H_2^0),\tag{SP:6.1.1--3}\\
%
\begin{split}
-\mathcal L\w{soft}&\supset
  m_{H\w d}^{2}|H\w d|^2
 + m_{H\w u}^{2}|H\w u|^2
 + (b\epsilon^{ab}H\w u^aH\w d^b +\text{h.c.}),
\end{split}\tag{SP:6.3.1}
%
% (|\mu|^2+{m^2_{H\w u}})|H_u|^2
%+(|\mu|^2+{m^2_{H\w d}})|H_d|^2
%+\left(b(H\w u^+H\w d^- - H\w u^0 H\w d^0)+\text{c.c.}\right),\tag{SP:8.1.1}
\end{align}
where $\epsilon^{12}=+1$ (2.13).
So the parameters are identified by the replacement
\begin{align}
\mu&=\mu,\\
m_3^2&=b,
\\
m_{H_1}^2 &= m_{H\w d}^2,
\\
m_{H_2}^2 &= m_{H\w u}^2,
\end{align}
where the LHS are the SLHA parameters and the RHS are those in SUSY primer.
With this identification, we can confirm that the above-shown formulae agree with SUSY Primer's equations (8.1.8)--(8.1.11) and (8.1.19)--(8.1.22).



\subsection{Interaction terms}
Here, to simplify the notation, we omit the SU(2) indices with assuming
\begin{equation*}
 AB=-BA\equiv A^1B^2-B^1A^2.
\end{equation*}
\subsubsection{SLHA convention}
The SLHA~\cite{SLHA} convention for the interection terms are
\begin{align}
 W &=
 - H_2 QY_U\bU
 + H_1 QY_D\bD
 + H_1 LY_E\bE,
\tag{SLHA:3}\\
 V_3 &=
 - H_2 \tilde Q T_U \tilde u^*
 + H_1 \tilde Q T_D \tilde d^*
 + H_1 \tilde L T_E \tilde e^*
\tag{SLHA:5}\\
 V_2 &=
  \tilde Q^*m_{Q}^{2} \tilde Q
 + \tilde L^*m_{L}^{2} \tilde L
 + \tilde u  m_{u}^{2} \tilde u^*
 + \tilde d  m_{d}^{2} \tilde d^*
 + \tilde e  m_{e}^{2} \tilde e^*
\tag{SLHA:7}\\
 \mathcal L_G &= \frac{M_1}{2}\tilde b\tilde b+\frac{M_2}{2}\tilde w\tilde w+\frac{M_3}{2}\tilde g\tilde g + \text{h.c.}
\tag{SLHA:9}
\end{align}


\subsubsection{Comparison to SUSY Primer convention}
In SUSY primer, the interaction terms are defined by
\begin{align}
 W\w{MSSM}
&\supset \bar U y_u Q H\w u - \bar d y_d Q H\w d - \bar e y_e L H\w d
\tag{SP:6.1.1}\\
&=
 -H_2Q y_u\tra \bar U + H_1 Qy_d\tra\bar d  +H_1 Ly_e\tra\bar e
\notag\\
V_3 &=
 \tilde u^* a_u \tilde Q H\w u
-\tilde d^* a_d \tilde Q H\w d
-\tilde e^* a_e \tilde L H\w d
\notag\\
&=
-H_2 \tilde Q a_u\tra \tilde u^*
+H_1 \tilde Q a_d\tra \tilde d^*
+H_1 \tilde L a_e\tra \tilde e^*
\notag\\
V_2 &\supset
 \tilde Q^* m_Q^2 \tilde Q
+\tilde L^* m_L^2 \tilde L
+\tilde u^* m_u^2 \tilde u
+\tilde d^* m_d^2 \tilde d
+\tilde e^* m_e^2 \tilde e
\notag\\
&=
 \tilde Q^* m_Q^2 \tilde Q
+\tilde L^* m_L^2 \tilde L
+\tilde u (m_u^2)\tra \tilde u^*
+\tilde d (m_d^2)\tra \tilde d^*
+\tilde e (m_e^2)\tra \tilde e^*
\notag\\
\mathcal L&\supset
-\frac{M_1}{2}\tilde b\tilde b-\frac{M_2}{2}\tilde w\tilde w-\frac{M_3}{2}\tilde g\tilde g + \text{h.c.}.\tag{SP:8.1.1}
\end{align}
So the notations can be matched with
\begin{align}
 Y_U, Y_D, Y_E &= y_u\tra, y_d\tra, y_e\tra\\
 T_U, T_D, T_E &= a_u\tra, a_d\tra, y_e\tra\\
 m_Q^2, m_L^2  &= m_Q^2, m_L^2\\
 m_u^2, m_d^2, m_e^2  &= (m_u^2)\tra,(m_d^2)\tra,(m_e^2)\tra,\\
 M_1, M_2, M_3 &= -M_1, -M_2, -M_3,
\end{align}
where the LHS are the SLHA parameters and the RHS are those in SUSY primer.


\end{document}
